\documentclass[journal]{IEEEtran}


\usepackage[czech]{babel}


% DOPLNIJICI BALICKY
\usepackage[utf8]		%	Kódování zdrojových souborů je Windows-1250
	{inputenc}					% Balíček pro nastavení kódování zdrojových souborů
\usepackage{cmap} 		% Balíček cmap zajišťuje, že PDF vytvořené `pdflatexem' je
											% plně "prohledávatelné" a "kopírovatelné"

\usepackage{amsmath}
\usepackage{caption}
\usepackage{dsfont}
\usepackage{layouts}
\usepackage{graphicx}
\usepackage{hyperref}
\usepackage{subcaption}
\usepackage{acro}

\include{ac}


\def\contentsname{Contents}
\def\listfigurename{List of Figures}
\def\listtablename{List of Tables}
\def\refname{References}
\def\indexname{Index}
\def\figurename{Obr.}
\def\tablename{Tab.}
\def\partname{Part}
\def\appendixname{Appendix}
\def\abstractname{Abstract}
% IEEE specific names
\def\IEEEkeywordsname{Klíčová slova}
\def\IEEEproofname{Proof}



% zkratky
\DeclareAcronym{MSE}{
  short=MSE,
  long=mean squared error
}

\DeclareAcronym{PSNR}{
  short=PSNR,
  long=peak signal-to-noise ratio
}

\DeclareAcronym{SSIM}{
  short=SSIM,
  long=structural similarity
}
% konec zkratek


\title{Název vašeho projektu - článku}


\author{Jmeno Prijmeni, Jmeno Prijmeni
        \linebreak
        Faculty of Informatics and Management
        \linebreak
        University of Hradec Kralove,
        \linebreak
        Hradec Kralove, Czech Republic
        \linebreak
        dobromi1@uhk.cz, dobromi1@uhk.cz

}

\begin{document}

% make the title area
\maketitle

% As a general rule, do not put math, special symbols or citations
% in the abstract or keywords.
\begin{abstract}
abstraktem se rozumí 10 až 15 řádků popisujících stručně obsah vašeho článku. Nejprve popište obecnou problematiku vašeho projektu, následně popište vámi řešený problém a pak čeho jste dosáhli a výsledky spolu s oblastí nasazení/použití.
\end{abstract}

% Note that keywords are not normally used for peerreview papers.
\begin{IEEEkeywords}
klíčové slovo 1; klíčové slovo 2; klíčové slovo 3; klíčové slovo 4; klíčové slovo 5
\end{IEEEkeywords}


\IEEEpeerreviewmaketitle



\section{INTRODUCTION/ÚVOD}

\subsection{OBSAH}
V úvodu definovat oblast bádání (problematiky) a postupně se dostat k tomu, kde je problém (v závěrečné části úvodu).
\subsection{ROZSAH}
Tato kapitola by měla mít rozsah minimálně do konce první strany dokumentu. Maximálně však do poloviny levého sloupce na druhé straně.
\subsection{CITACE}
V této kapitole budou alespoň 3, lépe však 5 odkazů na literaturu, vztahující se k popisované problematice, aby bylo z textu patrné, že se jedná o aktuální téma.
Minimálně je třeba nalézt 3 články z impaktovaných časopisů ISI WOK JCR, nebo z konferencí indexovaných ISI WOK CPCI. Najdete na stránce www.isiknowledge.com ale pouze pokud jste v rámci UHK (tedy buď na PC v budovách UHK, nebo přes VPN).



\section{PROBLEM DEFINITION/ DEFINICE PROBLÉMU}
\subsection{OBSAH}
V této kapitole je třeba definovat problém a ukázat alespoň tři řešení (lépe 5) od někoho jiného (formou odstavce shrnujícího přístup dotyčného (3 až 5 řádků)). Kapitola by měla končit konstatováním, že žádný z přístupů neřeší definovaný problém tak, jak by bylo třeba (jak bychom potřebovali my) a proto je třeba najít nový způsob (ten náš), o kterém se bude pojednávat v další kapitole.
\subsection{ROZSAH}
Tato kapitola by měla mít rozsah cca 1 stranu.
\subsection{CITACE}
V této kapitole budou alespoň 3, lépe však 5 odkazů na literaturu, vztahující se k popisované problematice, aby bylo z textu patrné, že se jedná o aktuální téma.




\section{NEW SOLUTION / NOVÉ ŘEŠENÍ}

\subsection{OBSAH}
V této kapitole je třeba přesně popsat nový způsob řešení a to včetně nutné teorie, která s tím souvisí. 
\subsection{ROZSAH}
Rozsahem je minimálně 1 strana a max. 2 strany.




\section{IMPLEMENTATION / IMPLEMENTACE ŘEŠENÍ}
\subsection{OBSAH}
Tato kapitola by měla pojednávat o praktické implementaci nového řešení. Tedy jak dojít od teorie k implementaci a jak jsme to řešili my (vy).
\subsection{ROZSAH}
Rozsah je min. 1 strana, maximálně 2 strany.





\section{TESTING OF DEVELOPED APPLICATION / TESTOVÁNÍ VYVINUTÉ APLIKACE - ŘEŠENÍ}
\subsection{OBSAH}
Zde musí být definice, jak bude testováno a co má být přesně výsledkem.
Vlastní testování a výsledky formou tabulek budou v podkapitole
Zhodnocení výsledků testování je nejlépe slovně (zhodnocení předchozích tabulek) a pak jedna tabulka s přehledem řešení od jiných autorů s tím novým řešením (mělo by se ukázat, že to nové řešení je nejlepší)
\subsection{ROZSAH}
Rozsah je 1strana.



\section{CONCLUSIONS / ZÁVĚRY}
Tady už se vyjádřit jen k tomu, že se podařilo najít (definovat) nový přístup k řešení problému a že byl i prakticky ověřen na modelovém případě. 
Dobré je také diskutovat využitelnost nového řešení jak v aktuální oblasti problému (nejlépe včetně finančních či časových úspor), tak i v dalších oblastech (alespoň nastínit).
Rozsah závěru je minimálně 10 řádků, maximálně 20 řádků.


% seznam zdrojů
\begin{thebibliography}{1}
\bibitem{}
Jahan, S., Begum, K., Shaheen, N., \& Khandokar, M. (2006). Near-Miss/Severe acute maternal morbidity (SAMM): A new concept in maternal care. Journal of Bangladesh College of Physicians and Surgeons, 24(1), 29-33.

\bibitem{}
World Health Organization, \& Unicef. (1996). Revised 1990 estimates of maternal mortality: a new approach by WHO and UNICEF

\end{thebibliography}

% nebo

% \bibliographystyle{plain}
% \bibliography{references.bib}{}


% that's all folks
\end{document}